\documentclass[m_shidai]{subfiles}
\begin{document}
{\Huge 釈迦}\\[5truemm]
\HUGE
\noindent
%\hspace{0.2zw}
\karifu{天\syotenvi{地}{4==\heidaku}\tsume 此}
{\kak\moji(-0.3,0)[n]{\parbox{2zw}{山ノキキ}}\moji[w]{ン}%天
\kaigyo{2.5}\呂由{\c}\呂由{\c}\chi%地
\kaigyo{2.5}\呂由{\c}\呂由{\c}\chi\modori\chi\hikidashi(-1,-0.5)[s]{切音不切息}\chikara{\c}\kak}%此
\let\temp=\hq\def\hq{25}
\karifu(1.5,-0.6)[c]{界}
{\kak\moji(-0.5,0)[s]{置声}\kili\kak\moji(-0.3,0)[s]{イ}\modori
            \shiyu{\k}\sho\kili\yurugu{\s}
	    \moveTo(-1,2.6)\yurikake{\hq}\kili\yurisori{\hq}
	    \kiri\woo\modori\呂由{\hq}\呂由{\hq}
	    \hqyu\kiri\呂由{\hq}\呂由{\hq}\hqyu}
\let\hq=\temp
\kugiri%{\small ○}
\hspace{0.2zw}
\karifu{\furichu{多}{助}{}聞室}
{\mawasu[1]{\c}\moji(-0.3,0)[s]{マワス}\modori\chi%多
\kaigyo{4}\jige\kili\woo\modoli\aki[0.1]{\hq}\chi\moji(-0.3,0)[ne]{ン}%聞
\kaigyo{1.5}\呂由{\c}\tsu}\\%室
\karifu[t]{\syotenvi{逝}{6==\maruten}\tsume\tsume
\syotenvi{宮}{1==\maruten}\tsume
\syotenvi{天}{1==\maruten}
\syotenvi{處}{6==\maruten}}
{\sho\modori\kak\modori\chi\moji[w]{イ}%逝
\kaigyo{3}\chi\modori\woo\kili\woo\modori\woo%宮
\kaigyo(1,2)\woo\kili\woo\modori\woo\moji[ne]{ン}%天
\kaigyo(1,3)\chikara{\w}\moji[ne]{ウ}%處
\呂由{\c}\modori{-110}[p]\chi\kak\modori\kak}%
\tsume\tsume
\karifu{
\syotenvi{十}{3==\heidaku}\tsume
\syotenvi{方}{6==\sumimaru}\tsume
\syotenvi{無}{6==\maruten}}
{\kak\modori\chi\tsu%十
\kaigyo{3}\jige\kili\woo\modoli\aki[0.1]{\hq}\chi%方
\kaigyo{2}\呂由{\c}}%無
\newline
\karifu{\syotenvi{丈}{1==\heidaku}\tsume
\syotenvi{夫}{1==\heidaku}\tsume
\syotenvi{牛}{4==\heidaku}\tsume
\syotenvi{王}{4==\maruten}}
{\sho\modori\kak\moji[w]{ウ}\modori\chi\%
\kaigyo{3}\chi\modori\woo\kili\woo\modori\woo%
\kaigyo(1,2)\woo\kili\woo\modori\woo\moji[ne]{ン}%
\kaigyo(1,3)\chikara{\w}\moji[ne]{ウ}%
\呂由{\c}\modoli\chi\kak\modori\kak}%
\tsume\tsume
\karifu{
\syotenvi{大}{1==\heidaku}
\syotenvi{沙}{6==\maruten}\tsume
\syotenvi{門}{6==\maruten}}
{\kak\modori\chi\moji[nw]{イ}%大
\kaigyo{3}\jige\kili\woo\modoli\aki[0.1]{\hq}\chi%沙
\kaigyo{2}\呂由{\c}\moji[nw]{ン}}%門
\newline
\karifu{\syotenvi{尋}{1==\heidaku}\tsume
\syotenvi{地}{1==\heidaku}\tsume
\syotenvi{山}{1==\maruten}
林}
{\sho\modori\kak\moji[w]{ム}\modori\chi%
\kaigyo{3}\chi\modori\woo\kili\woo\modori\woo%
\kaigyo(1,2)\woo\kili\woo\modori\woo\moji[ne]{ン}%
\kaigyo(1,3)\chikara{\w}\moji[ne]{ム}%
\呂由{\c}\modoli\chi\kak\modori\kak}%
\karifu{遍無}
{\moji[se]{ム}\呂由{\c}\kak%遍
\kaigyo{4}\moji[ne]{マワス}\mawasu*{-45}[1]{\c}%
\irohane{\w}\moji(0,0.3)[w]{ハヌル}}%無
\karifu(1,-0.3)[t]{等}{%\kaigyo(5,6)
\sori{\w}\律由{\k}\moji[6]{ウ}\律由{\k}%
\iromodori\sori{\w}\kak\kili\moveTo(1,0.5)\律由{\k}\律由{\k}\sho\kiri%
\chikara{\q}\modori\yurugu{\s}}%等
\newline
\karifu{香}{\sho}
\karifu(0,-0.5)[b]{華}{\kak\modori\chi\kak\modori\kak\moji[4]{イ}}
\karifu(0,-0.5)[c]{供}{\kak\modori\chi\矢[ne]}
\karifu(0,-0.5)[b]{養}
{\yurikake{\c}\moji(-1,0)[sw]{ウ}\kili\yurisori{\c}%
\modori[p]\woo\長[e]}
\karifu(0,-0.5)[b]{\furichu{\syotenvi{佛}{3==\maruten}}{二葉}{}}
{\呂由{\c}\woo\自下\kili[sw]\woo\modoli\aki[0.1]{\hq}\呂由{\c}%
\chi\tsu\moji(0,.2)[ne]{入}}
\end{document}