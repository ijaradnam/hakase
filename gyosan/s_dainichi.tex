\documentclass[m_shidai]{subfiles}
\begin{document}%
\HUGE
{\Huge 大日}\\
\def\gscale{0.4}
\karifu{歸}{\moji[sw]{キ}\chi\矢[ne]}
\karifu(0,0)[b]{命}{\呂由{\c}\moji(0,0)[sw]{ウ}\呂由{\c}\呂由{\c}\chi}
\karifu(0,-0.5)[c]{毘}{\呂由{\c}\呂由{\c}\chi\modori\chi%
            \hikidashi(-1,-0.5)[s]{切音不切息}\chikara{\c}\kak}
\karifu(1,-1.5)[c]{盧}{\kak\moji(-0.5,0)[s]{置声}\kili\kak\modori
            \shiyu{\k}\sho\kili\yurugu{\s}
	    \moveTo(-1,2)\yurikake{\hq}\kili\yurisori{\hq}
	    \kiri\woo\modori\呂由{\hq}\呂由{\hq}
	    \hqyu\kiri\呂由{\hq}\呂由{\hq}\hqyu}\\
%
\karifu(0,-0.5)[b]{\furichu{舎}{助}{}}{\mawasu[1]{\c}\moji(-0.3,0)[s]{マワス}\modori\chi}
\karifu(0,-0.4)[b]{那}{\自下\kili\woo\modoli\chi\矢[ne]}
\karifu(0,-0.4)[c]{\syotenvi{佛}{3==\heidaku}}{\呂由{\c}\tsu}
\karifu[b]{身}{\sho\modori\kak\moji(0,0)[w]{ム}\modori\chi\矢[ne]}
\karifu(0,-0.5)[c]{口}{\chi\modori\woo\長[e]\kili\woo\矢[w]
            \modori\woo\moji(-0.3,0)[e]{長}}\\
%
\karifu(0,-0.5)[b]{意}{\woo\長[e]\kili\woo\矢[w]
            \modori\woo\moji(-0.3,0)[e]{長}}
\karifu(0,-0.5)[c]{業}{\moji(0,0)[e]{ゴ}\chikara{\w}\moji(0,0)[w]{ウ}
            \呂由{\c}\modoli\chi\kak\矢[s]\modori\kak\長[n]}
\karifu(0,-0.5)[c]{遍}{\kak\modori\chi\矢[ne]\moji(0,0)[nw]{ン}}
\karifu(0,-0.5)[b]{虚}{\jige\kili[sw]\woo\modoli\chi\矢[ne]}
\karifu(0,-0.5)[c]{空}{\呂由{\c}}\\
%
\karifu(0,-0.5)[b]{演}{\sho\modori\kak\modori\chi\moji(0,0)[nw]{ン}}
\karifu(0,-0.5)[c]{説}{\chi\modori\woo\長[e]\kili\woo\矢[w]
            \modori\woo\長[e]\tsu\moji(0,0.2)[w]{入}}
\karifu(0,-0.5)[c]{如}
       {\woo\長[e]\kili\woo\矢[w]\modori\woo\長[e]}
\karifu(0,-0.5)[b]{来}
       {\chikara{\w}\moji(0,0.2)[ne]{イ}\呂由{\c}\modoli\chi
        \kak\矢[s]\modori\kak}
\karifu(0,-0.5)[c]{三}{\kak\modori\chi\矢[ne]}\\
%
\karifu(0,-0.5)[c]{密}
       {\jige\kili[sw]\woo\modoli\chi\tsu\moji(0,0.2)[w]{入}}
\karifu(0,-0.5)[b]{門}{\呂由{\c}\moji(0,0)[nw]{ン}}
\karifu(0,-0.5)[b]{金}
       {\moji(0,0)[ne]{コ}\sho\modori\kak\moji(0,0)[sw]{ム}%
	\modori\chi\矢[ne]}
\karifu(-0.2,-0.5)[c]{剛}
       {\chi\moji(0,0)[w]{ウ}\modori\woo\kili\woo\modori\woo}
\karifu(-0.2,-0.5)[c]{一}
       {\woo\kili\woo\modori\woo\tsu\moji(0,0.2)[w]{入}}\\
%
\karifu(0,-0.5)[b]{乗}
       {\chikara{\w}\moji(0,0)[w]{ウ}\呂由{\c}\modoli\chi%
        \kak\矢[s]\modori\kak\長[4]}
\karifu(0,-0.5)[b]{甚}{\moji(0,0)[sw]{ム}\呂由{\c}\modoli\chi\kak}
\karifu(0,-0.5)[b]{深}{\自下\moji(0,0)[sw]{ム}\irohane{\w}}
\karifu(1,-0.5)[t]{教}
       {\moji[s]{ケ}\sori{\w}\律由{\k}\moji(0,0)[2]{ウ}\律由{\k}%
        \iromodori\sori{\w}\kak\kili%
	\moveTo(1,0.5)\律由{\k}\律由{\k}\sho\kiri%
	\chikara{\q}\modori\yurugu{\s}}\\
%
\karifu{香}{\sho}
\karifu(0,-0.5)[b]{華}{\kak\modori\chi\kak\modori\kak\moji(0,0)[6]{イ}}
\karifu(0,-0.5)[b]{供}{\kak\modori\chi\矢[ne]}
\karifu(0,-0.5)[b]{養}{\yurikake{\c}\moji(-1,0)[7]{ウ}\kili\yurisori{\c}%
            \modori\woo\長[e]}
\karifu(0,-0.5)[b]{\furichu{\syotenvi{佛}{3==\maruten}}{二葉}{}}
           {\呂由{\c}\woo\自下\kili[sw]%
            \woo\modoli\呂由{\c}\chi\tsu\moji(0,.2)[ne]{入}}\\
\end{document}