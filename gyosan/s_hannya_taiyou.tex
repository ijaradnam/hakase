\documentclass[m_shidai]{subfiles}
\begin{document}
%對揚\\[5truemm]
\HUGE
\noindent
%{\LARGE 教主句}\\
\karifu[c]{南無十六}
{\qyu%南
\aki\aki\tsuya{\s}%無
\kaigyo[2]\moji[e]{ジ}\tsuya{\q}\moji(-0.3,0)[w]{ウ}%十
\aki\aki\tsuya{\q}\moji(-0.3,0)[w]{ク}}%六
\karifu(0,-0.3)[b]{會中}
{\iro{\q}\lwoo%會
\kaigyo{2}\iro{\c}\sho\moji[s]{ウ}}%中
\karifu[c]{\furichu{般}{立}{}若}
   {\chikara{\s}\moji[s]{ム}\cii\矢[n]%三
    \kaigyo{3}\律由{\c}\律由{\c}\kak\矢[s]\modori\kak\長[n]}%密
\newline\hfill
\karifu[b]{\furichu{教}{居}{}主}
   {\chikara{\s}\moji[w]{ウ}\tsuya{\q}
   \aki\moji[e]{シュ}\tsuyamochi{\q}\lwoo}
\karifu(-0.2,0)[t]{釈}{\chikara{\lw}}%\aki%釈
\karifu{迦}{\hiku{\s}\iro{\q}\lwoo\aki[0.3]{\q}\modori{-170}[n]%
    \sho\moji(-0.3,0)[nw]{ハル}}%迦
\karifu(0,-0.5)[t]{尊}{%\kaigyo{0.5}
   \uchitsuke{\q}\tsuya{\q}\moji[e]{ン}}\\[5truemm]
%尊
%\newline
%{\LARGE 證誠句}\\%十方擁護諸大神王
\karifu(-0.2,0)[c]{十方}{\chikara{\lw}\tsu%}
%\karifu(0,0.4)[c]{方}
   \aki{\q}\tsuyamochi{\q}\moji(-0.3,0)[w]{ウ}\sho}
\karifu(0,0)[c]{\furichu{擁\syotenvi{護}{3==\heidaku}}{以上一息}{}}
{\qyu\moveTo(-0.1,-0.1)\iro{\c}\sho
\kaigyo{2}\iro{\c}\sho\tsuya{\q}}
\karifu(0,0.3)[t]{諸\syotenvi{大}{1==\heidaku}}
  {\tsuya{\q}\ooyu{\q}\kili[e]
   \base(\s:1.0)\moji(-0.7,0)[se]{イ}\moji[sw]{{\LARGE 立}以下同}\modori*(\w:1.2)[p]\orisute
   \iro{\k}\sho\iro[2]{\k}\sho}
%\karifu(-0.5,0.3)[c]{\furichu{擁\syotenvi{護}{3==\heidaku}}{以上一息}{}}
%   {\iro{\c}\chikara{\s}\tsuya{\q}\aki\tsuya{\q}\tsu}
%\karifu(0,0.3)[c]{諸\syotenvi{大}{1==\heidaku}}
%   {\hiku{\s}\iro{\q}\lwoo\ooyu{\q}\kili[se]%
%    \base(\s:1.0)\moji[sw]{{\LARGE 立}以下同}\modori*(\w:1.2)[p]
%    \orisute\iro{\k}\sho\iro[2]{\k}\sho}%
\karifu(0,0)[t]{\furichu{神王}{居{\large 以下同}}{}}
    {\uchitsuke{\q}\tsuya{\q}\aki\tsuya{\q}}
\newline
%{\LARGE 神祇句}\\
\karifu[c]{天}{\tsuya{\q}\moji(-.3,0)[e]{ム}\sho}
\karifu(0,0.2)[t]{衆}{\moji[s]{ジュ}\iro{\c}\chikara{\s}\tsuya{\q}}
\karifu(0,0.2)[c]{\syotenvi{地}{6==\heidaku}}{\tsuya{\q}}
\karifu(0,0.5)[c]{類}{\tsuya{\q}\moji[e]{イ}}
\karifu{\syotenvi{倍}{6==\heidaku}%
\hspace{-0.5zw}\syotenvi{増}{1==\maruten}}
{\tsuya{\q}\moji[e]{イ}%倍
\ooyu{\q}\kili[se]\base(\s:1.0)\modori*(\w:1.2)[p]\orisute\iro{\k}\sho\iro[2]{\k}\sho}%増
\karifu(0,0)[t]{法楽}
{\uchitsuke{\q}\tsuya{\q}
\aki\tsuya{\q}\moji[e]{ク}}
\newline
%{\LARGE 證誠句}\\
\karifu(-0.2,0){當所}
{\chikara{\lw}\tsuya{\q}
\aki\qyu\moji(-0.5,0)[w]{口内当}}
\karifu(0,-0.2)[c]{鎮守}
{\qyu\moveTo(-0.1,-0.1)\iro{\c}\sho\moji[s]{ン}
\kaigyo{2}\iro{\c}\sho\tsuya{\q}}
\karifu(0,0.2)[c]{威光}
  {\tsuya{\q}\ooyu{\q}\kili[e]\moji(0.1,0)[se]{ウ}
   \base(\s:1.0)\modori*(\w:1.2)[p]\orisute
   \iro{\k}\sho\iro[2]{\k}\sho}
\karifu(0,0)[c]{\syotenvi{自}{6==\heidaku}\hspace{-0.3zw}\syotenvi{在}{6==\heidaku}}{\chikara{\s}\tsuya{\q}\aki\tsuya{\q}\moji[w]{イ}}\newline
%{\LARGE 霊句}\\%弘法大師倍増法楽
\karifu(-0.2,0){弘}{\chikara{\lw}\tsuya{\q}}
\karifu(0,0.3)[t]{法}{\qyu\moji(-0.5,0)[w]{口内当}}
\karifu[c]{大師}{\qyu\moveTo(-0.1,-0.1)\iro{\c}\sho\moji[s]{イ}%}
%\karifu(0,0.3)[c]{師}{
\kaigyo{2}\iro{\c}\sho\tsuya{\q}}
\karifu{\syotenvi{倍}{6==\heidaku}%
\hspace{-0.5zw}\syotenvi{増}{1==\maruten}}
{\tsuya{\q}\moji[e]{イ}%倍
\ooyu{\q}\kili[se]\base(\s:1.0)\modori*(\w:1.2)[p]\orisute\iro{\k}\sho\iro[2]{\k}\sho}%増
\karifu(0,-0.3)[t]{法楽}
{\uchitsuke{\q}\tsuya{\q}
\aki\tsuya{\q}\moji[w]{ク}}
\newline
%過去聖霊増進佛道
%\karifu(0.5,-0.3)[c]{\twarigaki{過 去法 界}}
%{\lwoo\aki{\q}\tsuya{\q}\sho}
%\karifu(0,-0.3)[c]{聖}{\qyu\moveTo(-0.1,-0.1)\iro{\c}\sho\moji[s]{イ}}
%\karifu(0,-0.3)[c]{霊}{\iro{\c}\sho\tsuya{\q}}
%%\karifu(0,-0.3)[c]{増}{\lwoo\tsuya{\q}}
%\karifu(0,-0.3)[c]{増進}{\lwoo\tsuya{\q}\ooyu{\q}\kili[se]\moji[e]{ウ}\base(\s:1.0)\modori*(\w:1.2)[p]\orisute\iro{\k}\sho\iro[2]{\k}\sho}
%\karifu(0,-0.3)[c]{佛道}{\sho\tsuya{\q}\tsu\moji[w]{入}\aki\tsuya{\q}\moji[e]{ク}}
%{\LARGE 聖朝句}\\%聖朝安穏国家豊楽
\karifu[c]{聖\ruby{朝}{テウ}}
{\lwoo\aki{\q}\tsuya{\q}\moji[e]{ウ}\sho}
\karifu(0,0.2)[c]{安穏}{\qyu\moveTo(-0.1,-0.1)\iro{\c}\sho\moji[s]{ン}%}
%\karifu(0,0.2)[c]{穏}{
\kaigyo{2}\iro{\c}\sho\tsuya{\q}\moji[w]{ン}}
\karifu[t]{国\ruby{家}{ケ}}
{\kaigyo{0.3}\lwoo\tsu\aki{\q}
\ooyu{\q}\kili[e]\moji(0.1,0)[se]{イ}\base(\s:1.0)\modori*(\w:1.2)[p]\orisute\iro{\k}\sho\iro[2]{\k}\sho}
\karifu(0,0)[c]
{\syotenvi{豊}{6==\heidaku}\hspace{-0.2zw}\syotenvi{楽}{3==\maruten}}
{\sho\tsuya{\q}\aki\tsuya{\q}\moji[e]{ク}}
\newline
%護持法主施主滅罪生善
%\karifu(0,-0.3)[c]{護持}
%{\lwoo\aki{\q}\tsuya{\q}\sho}
%%\karifu(0.5,-0.3)[c]{\twarigaki{法 主施 主}}{\qyu\moveTo(-0.1,-0.1)\iro{\c}\sho\moji[s]{イ}\kaigyo{3}\iro{\c}\sho\tsuya{\q}}
%\karifu(0.5,-0.3)[c]{\twarigaki{法施}}{\qyu\moveTo(-0.1,-0.1)\iro{\c}\sho}
%\karifu(0.5,-0.3)[c]{\twarigaki{主主}}{\iro{\c}\sho\tsuya{\q}}
%\karifu(0,-0.3)[c]{滅罪}{\lwoo\tsuya{\q}\ooyu{\q}\kili[se]\moji[e]{ウ}\base(\s:1.0)\modori*(\w:1.2)[p]\orisute\iro{\k}\sho\iro[2]{\k}\sho}
%\karifu(0,-0.3)[c]{生善}{\sho\tsuya{\q}\aki\tsuya{\q}}
%{\LARGE 伽藍句}\\%伽藍安穏興隆佛法
\karifu[c]{伽藍}
{\mawasu*{-45}[1]{\lw}\aki{\q}\qyu\moji[e]{ン}}
\karifu[c]{安穏}
{\qyu\moveTo(-0.1,-0.1)\iro{\c}\sho\moji[s]{ン}\nl{2}%
\iro{\c}\sho\tsuya{\q}\moji[w]{ン}}
\karifu[t]{興隆}
{\lwoo\moji[e]{ウ}\tsuya{\q}%興
\aki\moji[e]{リ}\ooyu{\q}\kili[e]\moji(0.1,0)[se]{ウ}\base(\s:1.0)%
\modori*(\w:1.2)[p]\orisute\iro{\k}\sho\iro[2]{\k}\sho}
\karifu(0,-0.5)[t]{佛法}
{\sho\tsuya{\q}\tsu\moji[w]{入}\aki\tsuya{\q}}
\newline
%{\LARGE 妙典句\\}
\karifu{演}{\lwoo}
\karifu(0,0.2)[c]{説}{\qyu}
%\karifu{演説}{\lwoo\aki{\q}\qyu}%
\karifu{甚深}{\qyu\moveTo(-0.1,-0.1)\iro{\c}\sho\moji[s]{ン}\kaigyo{2}%甚
\iro{\c}\sho\tsuya{\q}}%深
\karifu[t]{無相}{\tsuya{\q}\aki\ooyu{\q}\kili[e]\moji(0.1,0)[se]{ウ}\base(\s:1.0)\modori*(\w:1.2)[p]\orisute\iro{\k}\sho\iro[2]{\k}\sho}
\karifu(0,-0.5)[t]{了義}{\sho\tsuya{\q}\aki\tsuya{\q}}
\newline
%{\HUGe%
%{\LARGE 対告衆句}\\%證知証誠法涌菩薩
\newline
\karifu{\furichu{證知}{\large 以下四字早一息}{}}
{\chikara{\lw}\aki{\q}\aki{\q}\qyu\矢[e]}%證知
\karifu{証\syotenvi{誠}{6==\heidaku}}
{\aki\tsuyamochi{\q}\moji(-0.3,0)[e]{ウ}\sho%証
%\kaigyo{2}
\moveTo(0.5,-2)\moji[s]{ジョ}\iro{\c}\sho\tsuya{\q}\moji[sw]{ウ}}%誠
\karifu(0,0.2)[c]{法涌菩薩}%
{\qyu\矢[e]\aki\qyu\長[e]\律由{\c}%法涌
\kaigyo{3}\kak\moji[n]{急}%菩
\kaigyo{1}\kak\矢[s]\律由{\s}\tsu}%薩
%}
\end{document}