\documentclass[m_shidai]{subfiles}
\begin{document}%
%{\Huge 對揚}
{\HUGE
%\let\warisize=\wari
\noindent
\karifu[c]{南無}
{\qyu\moji(-0.6,0)[w]{少矢}%南
\aki\tsuya{\s}}%無
\karifu[c]{法界}
{\tsuya{\q}\tsu%法
\aki\hiku{\s}\iro{\q}\lwoo}%界
\karifu[b]{道場}{\qyu\長[w]\kaigyo{2}\iro{\c}\sho}
\karifu[c]{\furichu{三}{立}{}密}
{\chikara{\s}\moji(0,0)[sw]{ム}\chi\矢[n]%三
\kaigyo{3}\yuriR{\c}\yuriR{\c}\irotome{\k}\osu{\k}\sho\kak\矢[s]\tsu}\\%密
\hfill\karifu[b]{\furichu{教}{居}{}主}{\chikara{\s}\moji[w]{ウ}\tsuya{\q}\aki\moji[e]{シュ}\tsuyamochi{\q}\lwoo}
\karifu(-0.2,0)[c]{舎那尊}{\chikara{\lw}\aki[0.5]{\q}%舎
\hiku{\s}\iro{\q}\lwoo\modori{-170}[c]\sho\moji(-0.3,0)[nw]{ハル}%那
\kaigyo{0.5}\uchitsuke{\q}\tsuya{\q}\moji[w]{ン}}%尊

%四方四佛證誠加持
\karifu(-0.2,0.2)[c]{四方}{\chikara{\lw}\aki{\q}\tsuyamochi{\q}\moji(-0.3,0)[w]{ウ}\sho}
\karifu(0,0.3)[c]{\furichu{四\syotenvi{佛}{3==\heidaku}}{以上一息}{}}{\iro{\c}\chikara{\s}\tsuya{\q}\aki\tsuya{\q}\tsu}
\karifu(0,0.3)[c]{證\syotenvi{誠}{1==\heidaku}}{\hiku{\s}\iro{\q}\lwoo%
\ooyu{\q}\kili[se]%
\base(\s:1.0)\moji[sw]{{\LARGE 立}以下同}\modori*(\w:1.2)[p]\orisute\iro{\k}\sho\iro[2]{\k}\sho}%
\karifu(-0.2,0)[t]{\furichu{\twarigaki{密 教加 持}}{居{\large 以下同}}{}}
{\uchitsuke{\q}\tsuya{\q}\aki\tsuya{\q}}
\newline
%天衆地類倍増法楽
\karifu[c]{天}{\tsuya{\q}\moji(-.3,0)[e]{ム}\sho}
\karifu(0,0.2)[t]{衆}{\moji[s]{ジュ}\iro{\c}\chikara{\s}\tsuya{\q}}
\karifu(0,0.2)[c]{\syotenvi{地}{6==\heidaku}}{\tsuya{\q}}
\karifu(0,0.3)[c]{類}{\tsuya{\q}\moji[e]{イ}}
\karifu{\syotenvi{倍}{6==\heidaku}%
\hspace{-0.5zw}\syotenvi{増}{1==\maruten}}
{\tsuya{\q}\moji[e]{イ}%倍
\ooyu{\q}\kili[se]\base(\s:1.0)\modori*(\w:1.2)[p]\orisute\iro{\k}\sho\iro[2]{\k}\sho}%増
\karifu(0,0)[c]{法楽}
{\uchitsuke{\q}\tsuya{\q}
\aki\tsuya{\q}\moji[e]{ク}}
\newline
%當所鎮守威光自在
\karifu(-0.2,0)[c]{當所}
{\chikara{\lw}\tsuya{\q}
\aki\qyu\moji(-0.5,0)[w]{口内当}}
\karifu(0,0)[c]{鎮守}
{\qyu\moveTo(-0.1,-0.1)\iro{\c}\sho\moji[s]{ン}
\kaigyo{2.5}\iro{\c}\sho\tsuya{\q}}
\karifu(0,0)[c]{威光}
  {\tsuya{\q}\ooyu{\q}\kili[e]\moji[s]{ウ}
   \base(\s:1.0)\modori*(\w:1.2)[p]\orisute
   \iro{\k}\sho\iro[2]{\k}\sho}
\karifu(0,0)[c]{\syotenvi{自}{6==\heidaku}\hspace{-0.3zw}\syotenvi{在}{6==\heidaku}}{\chikara{\s}\tsuya{\q}\aki\tsuya{\q}\moji[w]{イ}}\newline
%弘法大師倍増法楽
\karifu(-0.2,0)[c]{弘}{\chikara{\lw}\tsuya{\q}}
\karifu(0,0)[t]{法}{\qyu\moji(-0.5,0)[w]{口内当}}
\karifu[c]{大}{\qyu\moveTo(-0.1,-0.1)\iro{\c}\sho\moji[s]{イ}}
\karifu(0,0)[c]{師}{\iro{\c}\sho\tsuya{\q}}
\karifu{\syotenvi{倍}{6==\heidaku}%
\hspace{-0.5zw}\syotenvi{増}{1==\maruten}}
{\tsuya{\q}\moji[e]{イ}%倍
\ooyu{\q}\kili[se]\base(\s:1.0)\modori*(\w:1.2)[p]\orisute\iro{\k}\sho\iro[2]{\k}\sho}%増
\karifu(0,0)[c]{法楽}
{\uchitsuke{\q}\tsuya{\q}
\aki\tsuya{\q}\moji[w]{ク}}
\newline
%過去聖霊増進佛道
\karifu(-0.4,0)[c]{\twarigaki{過 去法 界}}
{\lwoo\aki{\q}\tsuya{\q}\sho}
\karifu(0,0)[c]{聖}{\qyu\moveTo(-0.1,-0.1)\iro{\c}\sho\moji[s]{イ}}
\karifu(0,0)[c]{霊}{\iro{\c}\sho\tsuya{\q}}
\karifu(-0.2,0)[c]{増進}{\lwoo\tsuya{\q}\ooyu{\q}\kili[e]\moji[s]{ム}\base(\s:1.0)\modori*(\w:1.2)[p]\orisute\iro{\k}\sho\iro[2]{\k}\sho}
\karifu(0,-0.2)[c]{佛道}{\sho\tsuya{\q}\tsu\moji[w]{入}\aki\tsuya{\q}\moji[e]{ク}}
\newline
%聖朝安穏国家豊楽
\karifu[c]{聖\ruby{朝}{テウ}}
{\lwoo\aki{\q}\tsuya{\q}\moji[e]{ウ}\sho}
\karifu(0,0)[c]{安}{\qyu\moveTo(-0.1,-0.1)\iro{\c}\sho\moji[s]{ン}}
\karifu(0,0)[c]{穏}{\iro{\c}\sho\tsuya{\q}\moji[e]{ン}}
\karifu(-0.2,0)[c]{国\ruby{家}{ケ}}
{\lwoo\tsu\aki{\q}
\ooyu{\q}\kili[e]\moji[s]{イ}\base(\s:1.0)\modori*(\w:1.2)[p]\orisute\iro{\k}\sho\iro[2]{\k}\sho}
\karifu(0,0)[c]
{\syotenvi{豊}{6==\heidaku}\hspace{-0.2zw}\syotenvi{楽}{3==\maruten}}
{\sho\tsuya{\q}\aki\tsuya{\q}\moji[e]{ク}}
\newline
%護持法主施主滅罪生善
\karifu(-0.2,0)[c]{護持}
{\lwoo\aki{\q}\tsuya{\q}\sho}
\karifu(-0.1,0)[c]{\twarigaki{法施}}{\qyu\moveTo(-0.1,-0.1)\iro{\c}\sho}
\karifu(-0.1,0)[c]{\twarigaki{主主}}{\iro{\c}\sho\tsuya{\q}}
\karifu(-0.2,0)[c]{滅罪}{\lwoo\tsuya{\q}\ooyu{\q}\kili[e]\moji[s]{イ}\base(\s:1.0)\modori*(\w:1.2)[p]\orisute\iro{\k}\sho\iro[2]{\k}\sho}
\karifu(0,0)[c]{生善}{\sho\tsuya{\q}\aki\tsuya{\q}}
\newline
%伽藍安穏興隆佛法
\karifu[c]{伽藍}
{\mawasu*{-45}[1]{\lw}\aki{\q}\qyu\moji[e]{ン}}
\karifu[c]{安穏}
{\qyu\moveTo(-0.1,-0.1)\iro{\c}\sho\moji[s]{ン}\nl{2}%
\iro{\c}\sho\tsuya{\q}\moji[e]{ン}}
\karifu[c]{興隆}
{\lwoo\moji[e]{ウ}\tsuya{\q}%興
\aki\moji[e]{リ}\ooyu{\q}\kili[e]\moji[s]{ウ}\base(\s:1.0)%
\modori*(\w:1.2)[p]\orisute\iro{\k}\sho\iro[2]{\k}\sho}
\karifu[t]{佛法}
{\sho\tsuya{\q}\tsu\moji[w]{入}\aki\tsuya{\q}}
\newline
%法性無漏甚深妙典
%不空羂索薫入土砂
\karifu(-0.4,0)[c]{\furichu{\twarigaki{法 性 無 漏不 空 羂 索}}{\Large 理趣三昧所用}{}}
{\lwoo\tsuya{\q}\aki%法
\qyu\moji(-0.5,0)[w]{口内当}% 性
\aki[2]{\q}\qyu\moveTo(-0.1,-0.1)\iro{\c}\sho\moji[s]{イ}\kaigyo{2}% 無
\aki[0.4]{\hs}\iro{\c}\sho\tsuya{\q}}% 漏
\karifu(-0.4,-0.5)[b]{\twarigaki{甚 深薫 入}}
{\lwoo\tsuya{\q}\moji[se]{ニ}\ooyu{\q}\kili[e]\moji(0.1,0)[s]{ウ}\base(\s:1.0)\modori*(\w:1.2)[p]\orisute\iro{\k}\sho\iro[2]{\k}\sho}
\karifu(-0.4,0)[t]{\twarigaki{妙 典土 砂}}{\sho\tsuya{\q}\aki\tsuya{\q}\moji[e]{ク}}
\newline
{\HUGe%
\karifu{\furichu{所願成\syotenvi{辨}{6==\heidaku}}{\Large 以下四字少早一息}{}}
{\chikara{\lw}%所
\aki[0.5]{\q}\qyu\矢[w]\moji[s]{ン}\aki%願
\aki\tsuyamochi{\q}\moji(-0.3,0)[e]{ウ}\sho%成
\moveTo(0.4,-2)\moji[s]{ベ}\iro{\c}\sho\tsuya{\q}\moji[sw]{ン}}%辨
\karifu(0,0)[t]{金剛手菩薩}%
{\chikara{\q}\moji[s]{ム}\chi\矢[sw]\kaigyo{3}%金
\律由{\c}\moji[sw]{ウ}\kaigyo{3}%剛
\iro{\k}\sho\矢[nw]\kaigyo{2}%手
\kak\moji[n]{急}\moji(0.3,0)[w]{ソル勿れ}\kaigyo%菩
\kak\矢[s]\律由{\s}\tsu}%薩
}
}
\end{document}